%% Template for a preprint Letter or Article for submission
%% to the journal Nature.
%% Written by Peter Czoschke, 26 February 2004
%%

%\documentclass{nature}
\documentclass[a4paper,11pt]{article}
%% make sure you have the nature.cls and naturemag.bst files where
%% LaTeX can find them
% \bibliographystyle{naturemag}

\usepackage{epsfig}
\usepackage{subfigure}
\usepackage{amstext}
\usepackage{amsmath}
\usepackage{multicol}
\usepackage{pslatex}
\usepackage{apalike}
\usepackage{color}
\usepackage{caption}
\usepackage{graphicx}
\usepackage{hyperref}
\usepackage[margin=1in]{geometry}


\newcommand\dotaligner{\texttt{DotAligner}}
\newcommand\bralibase{\texttt{BRAliBase 2.1}}
\newcommand\pmcomp{\texttt{pmcomp}}
\newcommand\pmmulti{\texttt{pmmulti}}
\newcommand\clustgraph{\texttt{ClustGraph}}
\newcommand\locarna{\texttt{LocaRNA}}
\newcommand\foldalign{\texttt{FOLDALIGN}}
\newcommand\rnaplfold{\texttt{RNAplfold}}
\newcommand\carna{\texttt{CARNA}}
\newcommand\petfold{\texttt{PETfold}}	
\newcommand\rnadistance{\texttt{RNAdistance}}	
\newcommand\rnaalifold{\texttt{RNAalifold}}	
\newcommand\nw{\texttt{Needleman-Wunsch}}

\newenvironment{affiliations}{%
    \setcounter{enumi}{1}%
    \setlength{\parindent}{0in}%
    \slshape\sloppy%
    \begin{list}{\upshape$^{\arabic{enumi}}$}{%
        \usecounter{enumi}%
        \setlength{\leftmargin}{0in}%
        \setlength{\topsep}{0in}%
        \setlength{\labelsep}{0in}%
        \setlength{\labelwidth}{0in}%
        \setlength{\listparindent}{0in}%
        \setlength{\itemsep}{0ex}%
        \setlength{\parsep}{0in}%
        }
    }{\end{list}\par\vspace{12pt}}

\title{ Identification and classification of common RNA structure motifs }

%% Notice placement of commas and superscripts and use of &
%% in the author list

\author{Martin A. Smith$^{1,2,*\#}$, Stefan E. Seemann$^{2,3,*}$, XiuCheng Quek$^{1,2}$ \& John S. Mattick $^2$}


\begin{document}

\maketitle

\begin{affiliations}
\item Garvan Institute of Medical Research, 384 Victoria Street, Sydney 2010, Australia
\item  St Vincent’s Clinical School, UNSW Australia
\item University of Copenhagen, Groennegaardsvej 3, Frederiksberg, Denmark\\
 $^*$Contributed equally\\
 $^\#$Corresponding author\\
 m.smith[at]garvan.org.au 
\end{affiliations}


\begin{abstract}
\noindent  The abundance  and diversity of processed transcripts in eukaryotic genomes poses a challenge 
for the systemic classification of their biological functions. Sparse sequence conservation in 
non-coding sequences and the unreliable nature of RNA structure prediction algorithms further 
exacerbate this conundrum. Here, we present a unified computational approach for the 
unsupervised discovery and classification of homologous RNA structure motifs from a set of 
sequences of interest.  At its core lies DotAligner, a heuristic pairwise RNA structure alignment 
algorithm that considers both sequence similarity and the ensemble of sub-optimal RNA base pairings.  
Our approach outperforms other state of the art tools at classifying known RNA structure families, 
both in time and accuracy. When combined to density-based clustering using an empirically measured 
threshold, this method identifies both known and novel RNA structure motifs from ENCODE 
immuno-precipitation data for 44 proteins, further expanding the lexicon of functional transcriptomic motifs. 
\end{abstract}

\section{INTRODUCTION}
The human genome is pervasively transcribed into RNA with less than 2\%
encoding protein sequences. As genome technologies progress, an ever 
increasing amount of non-protein coding RNAs (ncRNA) are being discovered.
Once class of ncRNAs--long noncoding RNAs (lncRNAs)--are of particular 
interest for functional genome annotation given the large expanses they encompass. 
So far, only a relatively small quantity of lncRNAs have been functionally characterised, 
with regulation of gene expression and epigenetic states recurring as common 
biological functions \cite{morris2014rise,engreitz2016long}. Understanding the molecular mechanisms 
underlying the biological functions of lncRNAs--and how they are disrupted in disease--is
required to improve the functional annotation of the human genome. This is of
particular importance in the era of personalised genomics, as over 80\% of 
trait-associated single nucleotide polymorphisms occur in  non-coding regions 
\cite{hindorff2009potential,ritchie2014functional}.\\

The higher-order structure of RNA molecules is an essential feature of
non-coding RNAs that can be used for their classification and 
the inference of their biological function. Most small ncRNAs have well 
characterised secondary and tertiary structures, as evidenced by  RFAM, 
the largest collection of curated RNA families (2,588 families as of version 
12.2 \cite{rfam12}). However, determining the structural features of 
 long ncRNAs (lncRNAs) is a more complex problem given their
 size and, in general, faster evolutionary turnover. These challenges
 have raised doubts concerning the prevalence of functional structural 
 motifs in lncRNAs \cite{eddy2014computational,rivas2016statistical}, 
 despite evolutionary and biochemical support of conserved base-pairing 
 interactions \cite{smith2013widespread,spitale2015structural,lu2016rna}.\\

We, and others, hypothesise that lncRNAs act as scaffolds for the 
recruitment of proteins and assembly of ribonucleoproteins (RNPs), 
mediated by the presence of modular RNA structures, akin to the 
domain organisation of proteins 
\cite{zappulla2006rna,hogg2008structured,rinn2012genome,mercer2013structure,smith2013widespread,chujo2016architectural,blythe2016ins}. 
Protein-interacting regions of lncRNAs are likely to contain a combination 
of sequence and structure motifs that confer binding specificity. 
The higher-order structural features of protein-binding RNAs are 
subjected to different evolutionary dynamics than sequence constrained 
regions, where selective pressures to preserve higher-order 
structures facilitate compensatory and covariating mutations 
\cite{pang2006rapid,smith2013widespread,johnsson2014evolutionary}.\\
  
It has been shown that if sequence similarity falls below 60\%--the 
`twilight zone' of multiple sequence alignment--sequence-centric approaches 
for the identification of RNA structures perform poorly  \cite{Gardner15860779}. 

In addition, competing structures and
suboptimal structures may support or even drive the functionality of an RNA
domain. Hence, methods are needed that find structural similarity independent
from sequence conservation and freed from one single optimal RNA secondary
structure.

For clustering of RNA domains a dissimilarity measurement of all pairs of query
structures is needed. The dissimilarity is described through a pairwise weighted
string alignment with arbitrary pairwise dependencies (for base pairings). The
Needleman-Wunsch algorithm solves the maximum weight string alignment
problem by dynamic programming in $O(N^2)$ by preserving the sequence order and
maximizing the similarity. The consideration of pairs of nucleotides in each
sequence that form intra-molecular interactions extends the problem to pairwise
dependencies among positions in each string. This problem variant is
MAX-SNP-hard. However, the problem can be attacked by intelligent heuristics
that avoid the examination of all possible alignment states.

Simultaneous alignment and folding \cite{sankoff85} is the acknowledged gold
standard to predict the consensus structure and alignment of a set of related
RNA sequences. Because the Sankoff algorithm is practically not applicable, the
pre-calculation of the structure ensemble of each sequence, e.g. basepair
probabilities in thermodynamically equilibrated RNA structure ensembles
\cite{McCaskill:1990}, is used by different methods to speed up the calculation
of structure-based alignments. 1The programs \pmcomp{} for pairwise and
\pmmulti{} for multiple alignments \cite{Hofacker15073017}, as well as
\locarna{} \cite{Will17432929} score the alignment based on the notion of a
common secondary structure. Despite of the usage of the basepair probability
matrices these methods extract the maximum-weight common secondary structure but
do not explicitely consider suboptimal structures in the alignment. The pairwise
alignment of basepair probability matrices (dot plots) has been first introduced
by \carna{} \cite{Palu2010,Sorescu2012}. \carna{} finds iteratively better
alignments with an effective constraint programming technique using a branch and
bound scheme (propagator).

Beside of \locarna{} and a method based on directed acyclic graph kernels
\cite{Sato18647390}, the alignment-free approach \clustgraph{}
\cite{Heyne22689765} has been used to cluster RNA structure in common domains.
Here, we propose an alternative heuristic for the pairwise weighted string
alignment with arbitrary pairwise dependencies that can deliver dissimilarity
scores of dot plots in time close to an Needleman-Wunsch alignment which makes
the approach applicable for clustering of large numbers of putative RNA domains.

\section{ Materials and Methods }
\section*{ Ensemble-guided pairwise RNA structure alignment} 

\begin{figure}
 \includegraphics[width=\textwidth]{figX}
\caption \textbf{{Schematic of a pairwise alignment with DotAligner.} 
  A dynamic  programming matrix is first filled first filled in based on sequence 
  similarity (top left--color intensity indicates cumulative sequence similarity). 
  A partition function over all pairwise sequence alignments is then calculated and 
  interrogated for structural similarities by stochastic backtracking. Considering 
  the ensemble of suboptimal secondary structures--represented by dot plots in the 
  upper right and lower left quadrants--effectively warps the optimal sequence alignment 
  path (top left, black outline) towards one that includes structural features 
  (striped blue cells).  In the bottom right, the optimal sequence alignment 
  and associated consensus secondary structure is contrasted to that produced 
  by DotAligner, exposing the common structural features hidden in the suboptimal 
  base pairing ensemble of both sequences. 
 }
\end{figure}

\subsection{ Benchmarking and parameter optimisation. }

The \dotaligner{} algorithm implements several theoretical parameters that first need to 
be tuned before being applied to biological sequence analysis.\textbf{ All combinations} of 
core parameters were tested on the 8,976 pairwise RNA structure alignments curated in the 
\bralibase{} reference dataset \cite{Wilm2006enhanced}. For each set of parameter combinations, 
the amount of alignments producing identical structural topologies to the reference alignment
was determined using \textit{RNAdistance}. The Structural Conservation Index (SCI), a robust 
measure of RNA structural alignment integrity, were  also calculated for all resulting alignments. 
Baseline parameters were then selected via a product rank of the 2 aforementioned metrics 
(\textbf{supplementary data?}).\\

\subsection{ Classification of RNA secondary structure families. }

%include a distribution of the pairwise sequence identities for bin classification in supplementary

This was achieved by sampling the entire collection of RFAM entries with published structures 
(extracted from the Stockholm file description fields as detailed in script \textit{getFastaFromRfamSeed.sh}). 
 
in a stochastic manner (using the script . , while ensuring that all sampled sequences respected constraints 
on their sequence composition. Specifically, we extracted a high Pairwise Sequence Identity (PSI) 
and a low PSI set, where any two sequences from the same set present greater than or less than, respectively, 
55\% PSI. 

\subsection{ Clustering of RNA structures }

We resampled 

\section{ Results }
\section*{ Fast and accurate classification of RNA structures  } 
We compared the quality of the alignments produced by \dotaligner{} with the results of
three state of the art RNA structure alignment algorithms: \carna{}, \foldalign{} and \locarna{} \textbf{(ADD REFS)}
(\textbf{Figure 1}). In all three qualitative metrics we assessed,  \dotaligner{} performed the 
worst on average. Interestingly, many of the pairwise structure alignments produced 
SCI scores above those from the \bralibase{} reference alignments. Unless the reference
alignments are poorly annotated (a possibility given that RFAM entries may have been automatically 
generated based on similarity to a covariance model), this comparison exposes the possible 
tendency of global optimization algorithms to overestimate the amount of paired bases in 
consensus RNA structure predictions. In this regard, \dotaligner{} stands apart as it reports this 
phenomenon less than the other surveyed tools (\textbf{Figure 1B}). 

\begin{figure}
 \includegraphics[width=\textwidth]{fig1}
 \caption {RNA structure alignment quality. (A) Matthews Correlation Coefficient; 
 (B) Structural Conservation Index difference between computed and reference alignments; 
 (C) Topological  edit distance between the \rnaalifold{} consensus of the computed alignment 
 and the and reference consensus.  }
\end{figure}

Despite its underwhelming performance in reproducing benchmark alignments, \dotaligner{} 
excels elsewhere. The intended application of this algorithm is the identification and
classification of RNA structural motifs from a large and diverse set of sequences of interest. 
Therefore, we evaluated the ability of \dotaligner{} to distinguish distinct RNA structure
 topologies from a heterogeneous sample of known 
 RNA structure families. We performed all versus all 
pairwise structure alignment of stochastically sampled RFAM sequences, which 
were selected with constraints on their sequence composition,  highlighting any 
sequence-dependent bias (see \textbf{Methods} and \textbf{Supplementary Figure 2}). Despite the seemingly
poor quality of its pairwise alignments, \dotaligner{} reproduces the known classification 
of RFAM structures as well or better than other pairwise RNA structure alignment tools
(\textbf{Figure 2A,B}). \\

The classification accuracy  of \dotaligner{} is most comparable 
to \carna{}, another ensemble-based structural alignment algorithm, but the latter
requires substantially more time to perform the comparisons (\textbf{Figure 2C}). 
Indeed, \dotaligner{} is faster that all surveyed algorithms, with the notable 
exception of the Needleman-Wunsch Algorithm (NWA), a sequence alignment 
algorithm that ignores secondary structure information. NWA performs 
surprisingly well at classifying RNA sequences, most likely due to the presence
of stretches of homologous sequence within biologically related RNAs. On 
larger datasets derived from experimental data, NWA would likely suffer from 
higher false positive rates given the repetitive nature of higher-eukaryote genomes.

\section{ De novo identification of homologous RNA structures in noisy samples }

\section{ Identifying protein-binding RNA motifs from eCLIP data }
We applied this clustering methodology to recently published Enhanced Cross-Linking 
and ImmunoPrecipitation (eCLIP) sequencing data from the ENCODE consortium [\textbf{REF}], 
which contains eCLIP peaks associated to 44 RNA binding proteins (RBPs). We submitted the
sequences from eLCIP peaks with strong, highly significant enrichment over background 
to the above-mentioned clustering strategy (see \textbf{Methods}). 

[initial clustering data results]. 

If a structural motif is targeted by a RBP, it is highly probably that the associated CLIPseq peaks 
do not encompass the entire sequence that forms the structural motif, since 
reverse transcription will terminate at the covalent bond formed between the 
protein and the RNA. Consequently, large RNA structures bound by RBPs are 
difficult to model without additional structure probing data. This experimental 
approach also introduces a 3' bias in the sequence-specificity of RBP binding sites 
given the nature of 3' to 5' reverse transcription required for cDNA synthesis.

To address these caveats, we surveyed the potential to form locally-stable
RNA secondary structures in the regions directly flanking 36,161 filtered eCLIP 
peaks (see \textbf{Methods}). XXX\% overlap a predicted RNA secondary structure, 
compared to XXX random peaks. 

\begin{figure}
 \includegraphics[width=\textwidth]{fig2}
 \caption { Classification of known RNA structures. 
 (A) Receiving Operator Characteristic (ROC) curves measuring the classification
 accuracy by contrasting the computed similarity matrices of each algorithm 
 to a binary classification matrix of RFAM  sequences (1 if same family; 
 0 if different). High PID =  56-95\% pairwise sequence identity; Low PID  = 1-55\%.
 (B) Area Under the Curve (AUC) of the ROC values with 95\% confidence intervals; 
 (C) Distribution of computation time (N.B. a fixed upper limit of 120 s was 
 imposed for \carna{}). 
 }
\end{figure}

  \begin{figure}
 \includegraphics[width=\textwidth]{fig3}
 \caption { Density based clustering of known  RNA structures. 
 }
\end{figure}


\section*{Discussion}

Given its relative speed and accuracy, \dotaligner{} can be used to generate larger 
(dis)similarity matrices for cluster analysis than other pairwise structure 
alignment algorithms, or at least produce them with reasonable computational 
power.  



% \begin{thebibliography}{1}
% \bibitem{dummy} Articles are restricted to 50 references, Letters
% to 30.
% \bibitem{dummyb} No compound references -- only one source per
% reference.
% \end{thebibliography}

\bibliographystyle{apalike} 
\bibliography{dotaligner}

%% Here is the endmatter stuff: Supplementary Info, etc.
%% Use \item's to separate, default label is "Acknowledgements"

\section{ Acknowledgments }



%%
%% TABLES
%%
%% If there are any tables, put them here.
%%

\begin{table}
\centering
\caption{This is a table with scientific results.}
\medskip
\begin{tabular}{ccccc}
\hline
1 & 2 & 3 & 4 & 5\\
\hline
aaa & bbb & ccc & ddd & eee\\
aaaa & bbbb & cccc & dddd & eeee\\
aaaaa & bbbbb & ccccc & ddddd & eeeee\\
aaaaaa & bbbbbb & cccccc & dddddd & eeeeee\\
1.000 & 2.000 & 3.000 & 4.000 & 5.000\\
\hline
\end{tabular}
\end{table}

\end{document}
